%\documentclass[wert.tex]{subfiles}
%\documentclass[14pt,a4paper,twoside]{report}	% Размер основного шрифта и формата листа
\documentclass[16pt,a4paper,twoside]{report}
\usepackage{xltxtra,xunicode,polyglossia}						% Используется для вывода логотипа XeLaTeX
\newfontfamily\russianfont{Liberation Sans}
\usepackage{subfiles}
\setdefaultlanguage{russian}				% Основной язык текста
\setotherlanguage{english}					% Дополнительный язык текста
\linespread{1}							% Межстрочный интервал выбран полуторным
\usepackage[left=2cm,right=2cm,vmargin=2cm]{geometry} % Отступы по краям листа %%% требование РИО
\usepackage{ragged2e} % Fully justified text
\usepackage[dvipsnames]{xcolor}
\usepackage{newverbs}
%\newverbcommand{\cverb}{\color{OliveGreen}}{}

\usepackage{titlesec, blindtext} % подключаем нужные пакеты
\newcommand{\hsp}{\hspace{20pt}} % длина линии в 20pt
% titleformat определяет стиль
\titleformat{\chapter}[hang]{\Huge\bfseries}{\thechapter\hsp{|}\hsp}{0pt}{\Huge\bfseries}

\bibliographystyle{ugost2008}

\usepackage{ marvosym }
\usepackage[unicode,pdfborder={0 0 0}]{hyperref}
\hypersetup{colorlinks=true,	linkcolor = {Black},	citecolor = {RoyalBlue},	urlcolor = {RoyalBlue} }

\usepackage{verbatim,indentfirst,cite,enumerate,float,amsmath,amssymb,amsthm,amsfonts}
\usepackage{graphicx,fontspec,subfigure,bm}
\newcounter{contnumeq}
\newcounter{contnumfig}
\newcounter{contnumtab}

\setcounter{contnumeq}{0}           % Нумерация формул: 0 --- пораздельно (во введении подряд, без номера раздела); 1 --- сквозная нумерация по всей диссертации
\setcounter{contnumfig}{0}          % Нумерация рисунков: 0 --- пораздельно (во введении подряд, без номера раздела); 1 --- сквозная нумерация по всей диссертации
\setcounter{contnumtab}{0}          % Нумерация таблиц: 0 --- пораздельно (во введении подряд, без номера раздела); 1 --- сквозная нумерация по всей диссертации

\graphicspath{
	{Chapter-1/Addition_of_angular_velocity/}
	{Chapter-1/Addition_of_motion/demo-1/}
	{Chapter-1/Inertia/demo-1/}
	{Chapter-1/Inertia/demo-2/}
	{Chapter-1/Movement_along_the_loop/}
	{Chapter-1/Movement_along_the_loop/}
	{Chapter-1/Newton_laws_of_mechanics/demo-1/}
	{Chapter-1/Newton_laws_of_mechanics/demo-2/}
	{Chapter-1/The_law_of_momentum_conservation/demo-1/}
	{Chapter-1/The_law_of_momentum_conservation/demo-2/}
	{Chapter-2/Center_of_mass/demo-1/}
	{Chapter-2/Center_of_mass/demo-2/}
	{Chapter-2/Gyroscopic_effect/demo-1/}
	{Chapter-2/Gyroscopic_effect/demo-2/}
	{Chapter-2/Moment_of_inertia/demo-1/}
	{Chapter-2/Moment_of_inertia/demo-2/}
	{Chapter-2/Strange_roll/}
	{Chapter-2/The_law_of_momentum_conservation/}
	{Chapter-3/Ball_collision/}
	{Chapter-3/Maxwell_pendulum/}
	{Chapter-3/Potential_barrier/}
	{Chapter-3/Rolling_cylinder/}
	{Chapter-3/Transition_of_energy/}
	{Chapter-4/Coriolis_force/}
	{Chapter-4/Foucault_pendulum/}
	{Chapter-4/Frictial_forces/demo-1/}
	{Chapter-4/Frictial_forces/demo-2/}
	{Chapter-4/Hooke_law/}
	{Chapter-4/Inertial_forces/demo-1/}
	{Chapter-4/Inertial_forces/demo-2/}
%	{Chapter-4/Inertial_forces/demo-3/}
			 }
		 
\begin{document}

\newpage
\setcounter{page}{1} % указывает с какой страницы начать нумерацию
\renewcommand{\contentsname}{\center{\textcolor{PineGreen}{Оглавление}}}
\renewcommand{\refname}{\center{\textcolor{PineGreen}{Список литературы}}}
\tableofcontents
\thispagestyle{empty}

\newpage
\thispagestyle{empty}
\begin{figure}
	\centering
	\includegraphics[width=1\linewidth]{tit1}
	\label{fig:tit1}
\end{figure}

\newpage
	\subfile{Chapter-1/Addition_of_angular_velocity/Addition_of_angular_velocity.tex} \newpage % Сложение угловых скоростей 
	\subfile{Chapter-1/Addition_of_motion/demo-1/Addition_speeds.tex} \newpage % Сложение движений
	\subfile{Chapter-1/Inertia/demo-1/Pull_out_a_sheet.tex} \newpage % Выдергивание листа бумаги из-под стакана
	\subfile{Chapter-1/Inertia/demo-2/Impulse_and_two_threads.tex} \newpage % Импульс силы. Гиря и две нити
	\subfile{Chapter-1/Movement_along_the_loop/Loop.tex} \newpage % Петля Нестерова	
	\subfile{Chapter-1/Newton_laws_of_mechanics/demo-1/Second_Law_two_carts.tex} \newpage % Второй закон Ньютона. Две тележки
	\subfile{Chapter-1/Newton_laws_of_mechanics/demo-2/Third_Law_Archimedes.tex} \newpage % Третий закон Ньютона на примере силы Архимеда	

\newpage
	\subfile{Chapter-2/Center_of_mass/demo-1/Center_of_mass.tex} \newpage % Отыскание центра масс
	\subfile{Chapter-2/Center_of_mass/demo-2/Сone.tex} \newpage % Двойной конус
	\subfile{Chapter-2/Gyroscopic_effect/demo-1/Gyroscope_on_a_rotating_platform.tex} \newpage % Гироскоп на вращающейся платформе
	\subfile{Chapter-2/Gyroscopic_effect/demo-2/Top.tex} \newpage % Прецессия Гироскопа. Волчок
	\subfile{Chapter-2/Moment_of_inertia/demo-1/Cross_Oberbeck_Pendulum.tex} \newpage % Крестообразный маятник Обербека
	\subfile{Chapter-2/Moment_of_inertia/demo-2/Rotation_axis.tex} \newpage % Свободные оси вращения
	\subfile{Chapter-2/Strange_roll/Strange_roll.tex} \newpage % Послушная и непослушная катушка
	\subfile{Chapter-2/The_law_of_momentum_conservation/Zhukovsky_is_bench.tex} \newpage % Скамья Жуковского

\newpage
	\subfile{Chapter-3/Ball_collision/Elastic_and_non-elastic_shock.tex} \newpage % Упругий и неупругий удар
	\subfile{Chapter-3/Maxwell_pendulum/Maxwell_pendulum.tex} \newpage % Маятник Максвелла
	\subfile{Chapter-3/Potential_barrier/Potential_barrier.tex} \newpage % Потенциальный барьер
	\subfile{Chapter-3/Rolling_cylinder/Rolling_cylinder.tex} \newpage % Потенциальный барьер
	\subfile{Chapter-3/Transition_of_energy/Transition_of_energy.tex} \newpage % Изгиб

\newpage
	\subfile{Chapter-4/Coriolis_force/Coriolis_force.tex} \newpage % Сила Кориолиса
	\subfile{Chapter-4/Foucault_pendulum/Foucault_pendulum.tex} \newpage % Маятник Фуко
	\subfile{Chapter-4/Frictial_forces/demo-1/Dry_and_fluid_friction.tex} \newpage % Крестообразный маятник Обербека
	\subfile{Chapter-4/Frictial_forces/demo-2/Dry_friction_force.tex} \newpage % Силы сухого трения
	\subfile{Chapter-4/Hooke_law/Hooke_is_law.tex} \newpage % Закон Гука
	\subfile{Chapter-4/Inertial_forces/demo-1/Plumbs_on_a_rotating_platform.tex} \newpage % Отвесы на вращающейся платформе
	\subfile{Chapter-4/Inertial_forces/demo-2/Parabolic_surface_of_a_rotating_fluid.tex} \newpage % Параболическая поверхность вращающейся жидкости

%	\subfile{Chapter-4/Inertial_forces/demo-3/Dropping_a_chain_from_a_spinning_disk.tex} \newpage % Сбрасывание цепочки с вращающегося диска
\end{document}