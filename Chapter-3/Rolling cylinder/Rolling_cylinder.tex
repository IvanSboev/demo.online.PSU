\documentclass[All.tex]{subfiles}
%%---------------------------%
%%---- Обычный файл      ----%
%%---------------------------%
%\sloppy
%\documentclass[14pt,a4paper,oneside]{extarticle}	% Размер основного шрифта и формата листа
%\usepackage{xltxtra}						% Используется для вывода логотипа XeLaTeX
%\usepackage{xunicode}						% Кодировка документа
%\usepackage{polyglossia}					% Загружает пакет многоязыковой верстки
%\newfontfamily\russianfont{Book Antiqua}
%%\setmainfont{Liberation Serif}						% Основной шрифт текста
%\setmainfont{Book Antiqua}
%\setdefaultlanguage{russian}				% Основной язык текста
%\setotherlanguage{english}					% Дополнительный язык текста
%\linespread{1}							% Межстрочный интервал выбран полуторным
%\usepackage[left=2.5cm,
%right=1.5cm,vmargin=2.5cm]{geometry} % Отступы по краям листа
%\bibliographystyle{ugost2008}
%
%\usepackage{xcolor}
%\usepackage{hyperref}
%% Цвета для гиперссылок
%\definecolor{linkcolor}{HTML}{359B08} % цвет ссылок
%\definecolor{urlcolor}{HTML}{799B03} % цвет гиперссылок
%\hypersetup{pdfstartview=FitH,  linkcolor=linkcolor,urlcolor=urlcolor, colorlinks=true}
%
%%---------------------------%
%%---- Пакеты расширений ----%
%%---------------------------%
%\usepackage{xcolor}
%\usepackage{hyperref}
%% Цвета для гиперссылок
%\definecolor{linkcolor}{HTML}{359B08} % цвет ссылок
%\definecolor{urlcolor}{HTML}{799B03} % цвет гиперссылок
%\hypersetup{pdfstartview=FitH,  linkcolor=linkcolor,urlcolor=urlcolor, colorlinks=true}
%
%
%\usepackage{verbatim,indentfirst}
%\usepackage{cite,enumerate,float}
%\usepackage{amsmath,amssymb,amsthm,amsfonts}
%
%%---------------------------%
%%--- Вставка иллюстраций ---%
%%---------------------------%
%\usepackage{graphicx}
%\usepackage{subfigure}
%%\graphicspath{{Images/}}
%\usepackage{fontspec}

\begin{document}
%	\pagestyle{empty} %  выключаенм нумерацию
	
	%\setcounter{page}{3}% Нумерация начинается с третьей страницы
	%\renewcommand{\contentsname}{\center{Содержание}}
	%\tableofcontents
	

		%\addcontentsline{toc}{section}{Потенциальный барьер}
		\section{Скатывание двух цилиндров}

		

\begin{figure}[H] 	
	\centering 	
	\includegraphics[width=0.9\linewidth]{inclinedplane-1.png}
	\caption{Демонстрация зависимости инертных свойств тел от распределения массы в этих телах на примере скатывания сплошного и полого цилиндров равной массы и одинакового размера с наклонной плоскости}
	\label{inclinedplane-1}
\end{figure}
	
	\subsection*{\textcolor{PineGreen}{Оборудование}}

		\begin{enumerate}
			\item Два цилиндра одинаковой массы
			\item Весы.
			\item Наклонная плоскость.
			\item Линейка или указка.
		\end{enumerate}
		
	\subsection*{\textcolor{PineGreen}{Краткое описание}}
		
	На наклонную плоскость кладут два цилиндра одинаковой массы и радиуса.
	Цилиндры располагают так, чтобы их оси были находились одна на продолжении другой и удерживают в равновесии.
	
	После отпускания цилиндров они скатываться одновременно с наклонной плоскости, при этом один обгоняет другой. Оказывается, что цилиндр, масса которого сосредоточена ближе к центру, движется с большим ускорением. Это объясняется тем, что его момент инерции оказывается меньше, чем у полого цилиндра, вся масса которого находится на значительном расстоянии от оси вращения.
	
	\begin{figure}[H] 	
		\centering 	
		\includegraphics[width=0.5\linewidth]{inclinedplane-2.png}
		\caption{При одновременном отпускании цилиндров быстрее будет скатываться тот, чей момент инерции окажется меньше. При одинаковых размерах и массе моменты инерции двух цилиндров (сплошной и полый) будут отличаться вдвое}
		\label{inclinedplane-2}
	\end{figure}

	Опыт позволяет наглядно продемонстрировать, что чем больше момент инерции, тем медленнее изменяется линейная скорость тел при одинаковом размере и равной массе.
	
	\subsection*{\textcolor{PineGreen}{Теория}}
	
	При описании движения цилиндрического тела с наклонной плоскости воспользуемся уравнением движения центра масс, а также основным законом динамики вращательного движения.
	


	В векторной форме уравнение поступательного движения центра масс цилиндра запишется следующим образом:
	\begin{equation}\label{inclinedplane-1eq1}
	m\textbf{g}+\textbf{N}+\textbf{F}_{\text{тр}} = m\textbf{a}.
	\end{equation} 
	
В выбранной системе координат после проектирования всех векторов можно записать уравнение движения в скалярном виде.
В проекции на ось  \textit{x}  это уравнение примет вид:
	\begin{equation}\label{inclinedplane-1eq2}
	mg\sin\alpha - F_{\text{тр}} = ma,
	\end{equation}
	в проекции на ось  \textit{z}:
	\begin{equation}\label{inclinedplane-1eq3}
	N - mg\cos\alpha = 0.
	\end{equation}
	
		\begin{figure}[H] 	
		\centering 	
		\includegraphics[width=0.6\linewidth]{inclinedplane-3.png}
		\caption{Схематичное изображение сил, действующих на цилиндр при его движении с наклонной плоскости. Сила трения создает вращательный момент, поэтому скатывающийся ускоренно цилиндр начинает закручиваться. Согласно основному закону динамики вращательного движения угловое ускорение точек цилиндра, а следовательно, и линейное ускорение его центра масс, оказывается тем больше, чем меньше его момент инерции}
		\label{inclinedplane-3}
	\end{figure}

	Составим основное уравнение вращательного движения относительно оси, проходящей через центр масс цилиндра. 
	Моменты силы тяжести  $	m\textbf{g} $ и реакции опоры $ \textbf{N} $ относительно этой оси равны нулю. 
	Угловое ускорение $ \textbf{ε} $ определяется только моментом силы трения $ F_{\text{тр}} $ и моментом инерции $ I $:
	\begin{equation}\label{inclinedplane-1eq4}
	 I\textbf{ε}=\textbf{M}
	\end{equation}
	где \textit{I} — момент инерции цилиндра относительно оси вращения, \linebreak $ \textbf{M} = \textbf{F}_{\text{тр}}\times \textbf{r} $ — момент силы трения, определяемый через векторное произведение силы трения на радиус-вектор.
	
	В проекции на ось $ y $ уравнение вращательного  движения (\ref{inclinedplane-1eq4}) примет вид:
	\begin{equation}\label{inclinedplane-1eq5}
	I\varepsilon = F_{\text{тр}} r.
	\end{equation}
	 
	Пользуясь известным соотношением между линейным и угловым ускорениями при движении без проскальзывания $ a = r\varepsilon $, выразим силу трения:
	\begin{equation}\label{inclinedplane-1eq6}
	F_{\text{тр}} = \frac{Ia}{r^{2}}.
	\end{equation}

	Подставляя найденную силу трения в уравнение движения (\ref{inclinedplane-1eq2}), получим:
		\begin{equation}\label{inclinedplane-1eq7}
		mg\sin\alpha -  \frac{Ia}{r^{2}} = ma.
	\end{equation}
	
	Отсюда можно выразить линейное ускорение \textit{a} центра масс скатывающегося цилиндра  

	\begin{equation}\label{8}
	a =  \frac{mg\sin\alpha}{I/r^{2} + m}  = \frac{g\sin\alpha}{1 + I/mr^{2}}.
	\end{equation} 
	  
	  Из полученного выражения следует, что изменение скорости твердого тела при движении по наклонной плоскости зависит от его момента инерции.
	 Увеличение момента инерции твердого тела приводит к уменьшению ускорения центра масс тела.
	 Таким образом, сплошной цилиндр, обладающий меньшим моментом инерции (вся его масса распределена вблизи оси вращения и момент инерции равен $ I_{1} = mr^{2}/2 $), скатывается быстрее, по сравнению с тонкостенным полым цилиндром, у которого масса в основном находится на периферии ($ I_{2} = mr^{2}  $).    
	
\end{document}
