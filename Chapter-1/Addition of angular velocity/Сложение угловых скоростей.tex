\sloppy
\documentclass[14pt,a4paper,twoside]{extarticle}	% Размер основного шрифта и формата листа
\usepackage{xltxtra}						% Используется для вывода логотипа XeLaTeX
\usepackage{xunicode}						% Кодировка документа
\usepackage{polyglossia}					% Загружает пакет многоязыковой верстки
\newfontfamily\russianfont{Book Antiqua}
%\setmainfont{Liberation Serif}						% Основной шрифт текста
\setmainfont{Book Antiqua}
\setdefaultlanguage{russian}				% Основной язык текста
\setotherlanguage{english}					% Дополнительный язык текста
\linespread{1}							% Межстрочный интервал выбран полуторным
\usepackage[left=2.5cm,
right=1.5cm,vmargin=2.5cm]{geometry} % Отступы по краям листа
\bibliographystyle{ugost2008}

\usepackage{xcolor}
\usepackage{hyperref}
% Цвета для гиперссылок
\definecolor{linkcolor}{HTML}{359B08} % цвет ссылок
\definecolor{urlcolor}{HTML}{799B03} % цвет гиперссылок
\hypersetup{pdfstartview=FitH,  linkcolor=linkcolor,urlcolor=urlcolor, colorlinks=true}

%---------------------------%
%---- Пакеты расширений ----%
%---------------------------%
\usepackage{xcolor}
\usepackage{hyperref}
% Цвета для гиперссылок
\definecolor{linkcolor}{HTML}{359B08} % цвет ссылок
\definecolor{urlcolor}{HTML}{799B03} % цвет гиперссылок
\hypersetup{pdfstartview=FitH,  linkcolor=linkcolor,urlcolor=urlcolor, colorlinks=true}


\usepackage{verbatim,indentfirst}
\usepackage{cite,enumerate,float}
\usepackage{amsmath,amssymb,amsthm,amsfonts}

%---------------------------%
%--- Вставка иллюстраций ---%
%---------------------------%
\usepackage{graphicx}
\usepackage{subfigure}
\usepackage{fontspec}
%\graphicspath{{Images/}}

\begin{document}
%	\pagestyle{empty} %  выключаенм нумерацию
%\setcounter{page}{3}% Нумерация начинается с третьей страницы
%\renewcommand{\contentsname}{\center{Содержание}}
%\tableofcontents

\newpage
\begin{center}
%	\addcontentsline{toc}{section}{Опыт 6. Сложение угловых скоростей}
	\subsection*{Сложение угловых скоростей}
\end{center}

\begin{figure}[H] 
	\centering 	
	\includegraphics[width=0.6\linewidth]{angular-1.png}
	\caption{Демонстрация сложения угловых скоростей на центробежной машине}
	\label{angular-1}
\end{figure}

\subsection*{\underline{Оборудование:}}

\begin{enumerate}
	\item Шар диаметром 35 см, покрытый по линиям широт рядом пятен белого цвета диаметром 1 см
	\item Вращающийся держатель
	\item Машина с червячным механизмом
\end{enumerate}

\newpage
\subsection*{\underline{Основные определения:}}

Угловая скорость — величина, характеризующая быстроту вращения твердого тела. 
При равномерном вращении тела вокруг неподвижной оси его угловая скорость численно равна приращению угла поворота $ \varphi $ за промежуток времени $ \Delta t $
$$ \omega = \Delta \varphi/ \Delta t. $$
 
В общем случае угловая скорость численно равна отношению элементарного угла поворота $d\varphi $ 
к соответствующему элементарному промежутку времени $ dt $, то есть $$ \omega = d\varphi/dt. $$ 

Таким образом,
\begin{flushleft}
	\textit{вектор угловой скорости} \textbf{ω} \textit{численно равен величине угловой скорости, лежит на оси вращения, и направление его связано с направлением вращения правилом буравчика}.
\end{flushleft}

Поскольку угловая скорость — вектор, то приращение ее также вектор и, следовательно, вектором является угловое ускорение:
$$ \textbf{ε} = \frac{d\textbf{ω}}{dt}.$$

Между векторами угловой и линейной скоростей существует связь.
\begin{flushleft}
	\textit{Вектор линейной скорости точки при вращательном движении равен векторному произведению вектора угловой скорости на радиус-вектор точки}
\end{flushleft}
$$ \textbf{v} = \textbf{ω}\times \textbf{r}. $$

\newpage
\subsection*{\underline{Краткое описание:}}

Шар закрепляется в специальном держателе, в котором он может вращаться вокруг наклонной оси, а вместе с держателем — вокруг вертикальной оси при помощи червячной машины.
Таким образом вращать шар можно вокруг наклонной, а затем вертикальной оси, либо одновременно вокруг обеих осей.

При быстром вращении шара вокруг наклонной оси пятна на его поверхности сливаются и образуют параллельные ряды (рис.\ref{angular-2},\textit{а})
Направление оси вращения шара совпадает с направлением угловой скорости вращения $ \textbf{ω}_{1} $.

\begin{figure}[H] 	
	\centering 	
	\includegraphics[width=0.9\linewidth]{angular-2.png}
	\caption{\textit{а} — вращение шара только вокруг наклонной (собственной) оси; \textit{б} — вращение шара только вокруг вертикальной оси}
	\label{angular-2}
\end{figure}

При вращении шара только вокруг вертикальной оси пятна сливаются в линии (рис.\ref{angular-2},\textit{б}), которые лежат в горизонтальной плоскости, а оси этих линий совпадают с осью вращения держателя.
Вектор $ \textbf{ω}_{2} $ в этом случае направлен вертикально.



\newpage
\subsection*{\underline{Теория:}}

Для перемещения, скорости и ускорения справедливы правила векторного 
сложения. 
Имея в виду эту справедливость векторного сложения, 
получается, что для механических движений справедлив принцип независимого сложения.
Этот принцип гласит, что отдельные движения, в которых участвует тело в данной системе отсчета, не влияют друг 
на друга, что всегда любое движение можно представить как сумму других независимых движений.

\begin{figure}[H] 	
	\centering 	
	\includegraphics[width=0.8\linewidth]{angular-3.png}
	\caption{\textit{а} — схематичное изображение шара на вращающемся стержне (без вращения); \textit{б} — вектор угловой скорости при вращении шара только вокруг собственной (наклонной) оси; \textit{в} — сложение угловых скоростей при одновременном вращении шара вокруг собственной и вертикальной осей}
	\label{angular-3}
\end{figure}

Исходя из преобразований Галилея можно сделать важный вывод о связи скоростей при сложном движении тела.
Для точек на поверхности шара, движущихся одновременно вокруг обоих осей, их результирующую линейную скорость можно представить в виде:

\begin{equation}\label{angular-eq1}
\textbf{v} = \textbf{v}_1 + \textbf{v}_2.
\end{equation}

Используя известную связь между линейной и угловой скоростями, можно получить следующее выражение:
\begin{equation}\label{angular-eq2}
\textbf{v} = \textbf{ω}_1 \times \textbf{r} + \textbf{ω}_2 \times \textbf{r} = (\textbf{ω}_1 + \textbf{ω}_2) \times \textbf{r} .
\end{equation}
где через \textbf{r} обозначен радиус-вектор, направленный из центра шара в рассматриваемую точку на его поверхности.

Следовательно, при одновременном вращении шара и держателя, результирующая угловая скорость  \textbf{ω} также будет представлять собой геометрическую сумму $ \textbf{ω}_1 $ и $ \textbf{ω}_2 $, и иметь направление, показанное на рис.\ref{angular-3},\textit{в}.
Это направление легко определить опытным путем так как кружки, расположенные вблизи мгновенной оси (вдоль $ \omega $), не сливаются в линии.

Таким образом, при одновременном вращении шара вокруг собственной оси и вокруг оси вращения держателя вектор результирующей угловой скорости, а, следовательно, и мгновенная ось вращения также поворачиваются вокруг вертикальной оси держателя.

\end{document}