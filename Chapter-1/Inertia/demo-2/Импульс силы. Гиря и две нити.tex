\sloppy
\documentclass[14pt,a4paper,oneside]{extarticle}	% Размер основного шрифта и формата листа
\usepackage{xltxtra}						% Используется для вывода логотипа XeLaTeX
\usepackage{xunicode}						% Кодировка документа
\usepackage{polyglossia}					% Загружает пакет многоязыковой верстки
\newfontfamily\russianfont{Book Antiqua}
%\setmainfont{Liberation Serif}						% Основной шрифт текста
\setmainfont{Book Antiqua}
\setdefaultlanguage{russian}				% Основной язык текста
\setotherlanguage{english}					% Дополнительный язык текста
\linespread{1}							% Межстрочный интервал выбран полуторным
\usepackage[left=2.5cm,
right=1.5cm,vmargin=2.5cm]{geometry} % Отступы по краям листа
\bibliographystyle{ugost2008}

\usepackage{xcolor}
\usepackage{hyperref}
% Цвета для гиперссылок
\definecolor{linkcolor}{HTML}{359B08} % цвет ссылок
\definecolor{urlcolor}{HTML}{799B03} % цвет гиперссылок
\hypersetup{pdfstartview=FitH,  linkcolor=linkcolor,urlcolor=urlcolor, colorlinks=true}

%---------------------------%
%---- Пакеты расширений ----%
%---------------------------%
\usepackage{xcolor}
\usepackage{hyperref}
% Цвета для гиперссылок
\definecolor{linkcolor}{HTML}{359B08} % цвет ссылок
\definecolor{urlcolor}{HTML}{799B03} % цвет гиперссылок
\hypersetup{pdfstartview=FitH,  linkcolor=linkcolor,urlcolor=urlcolor, colorlinks=true}


\usepackage{verbatim,indentfirst}
\usepackage{cite,enumerate,float}
\usepackage{amsmath,amssymb,amsthm,amsfonts}

%---------------------------%
%--- Вставка иллюстраций ---%
%---------------------------%
\usepackage{graphicx}
\usepackage{subfigure}
\usepackage{fontspec}
%\graphicspath{{Images/}}

\begin{document}
%	\pagestyle{empty} %  выключаенм нумерацию
%\setcounter{page}{3}% Нумерация начинается с третьей страницы
%\renewcommand{\contentsname}{\center{Содержание}}
%\tableofcontents

\begin{center}
	\subsection*{Импульс силы. Гиря и две нити}
\end{center}

\begin{figure}[H] 	% Окружение для вставки иллюстрации
	\centering 		% Выравнивание по центру
	\includegraphics[width=0.9\linewidth]{inertia-2.png}
	\caption{Демонстрация явления инерции}
\end{figure}

\subsection*{\underline{Оборудование:}}

\begin{enumerate}
	\item Штатив
	\item Гиря с крючками, ввинченными в верхний и нижний торцы
	\item Несколько нитей одинаковой толщины и длины
	\item Резиновый коврик, используемый для амортизации удара
\end{enumerate}

\newpage
\subsection*{\underline{Основные определения:}}

Величина, количественно определяющая те действия тел друг 
на друга, которые вызывают ускорения, называется силой. 
С одной стороны, сила есть количественная мера действий тел друг на друга.
С другой стороны, сила есть количественная мера тех действий, которые вызывают ускорения.  

Конечная скорость движения тел определяется не только самой 
силой, но и временем действия этой силы. 

Импульсом называется мера механического движения, равная для материальной точки произведению ее массы $ m $ на скорость $ \textbf{v} $.
Такая величина как импульс $ \textbf{p} = m\textbf{v} $ — является векторной, направленной так же, как скорость точки.
Импульс $ \textbf{P} $ механической системы равен геометрической сумме импульсов всех ее точек, или произведению массы $ М $ всей системы на скорость $ \textbf{V}_{c} $ её центра масс: $$ \textbf{P} = \sum m_{i} \textbf{v}_{i} =  M\textbf{V}_{c}. $$

При действии силы \textbf{F} импульс точки изменяется в общем случае и численно и по направлению; это изменение определяется вторым (основным) законом динамики.
Изменение импульса системы происходит под действием только внешних сил, то есть сил, действующих на систему со стороны тел, в эту систему не входящих.

Импульс силы \textbf{S} — это сложная физическая величина, которая 
одновременно учитывает влияние модуля, направления и времени 
действия силы на изменение состояния движения тела.
Импульс силы $\textbf{S} = \textbf{F}\Delta t $ является вектором, по направлению совпадающим с направлением вектора силы \textbf{F}. 
Импульс системы — величина векторная и направлен он в ту же сторону, что и вектор результирующей силы $ \textbf{F}_{p} $.

В новых понятиях второй закон Ньютона можно прочитать следующим образом: 
\begin{flushleft}
	\textit{изменение импульса тела равно импульсу всех сил, действовавших на него:}
\end{flushleft}

\begin{center}
		$\textbf{F}\Delta t = m\textbf{v}_{2} - m\textbf{v}_{1} $ или $\textbf{F}\Delta t = \Delta(m\textbf{v}) $
\end{center}

Именно в таком виде закон был впервые сформулирован самим И. Ньютоном 

\newpage
\subsection*{\underline{Краткое описание демонстрации:}}

К штативу, установленном на ровной поверхности, при помощи тонкой нити подвешивается тело (цилиндр массой $ 200 $ г).
Прочность нитей подбирается так, чтобы верхняя нить могла только удерживать тело, не разрываясь.
В нижнюю часть этого цилиндра ввинчивается крючок с привязанным к нему такой же нитью.
Для амортизации удара на стол кладется резиновый коврик.

Если во время опыта нижнюю нить резко с большой силой дернуть, то она разорвется, а верхняя как бы не «почувствует» сильного рывка (точка 2 на рис.\ref{inertia-3}). 
Причина кроется в том, что большая сила \textbf{F} действует на груз в течение очень короткого времени $ \Delta t $. 
Это время затрачивается только на создание деформации нижней нити во время рывка. 
Тело массой \textit{m} получает незначительный импульс силы, поэтому не успевает набрать скорость и сдвинуться с места.
По этой причине в верхней нити не возникают дополнительных деформации, и она оcтается целой. 

Если нижнюю нить потянуть плавно с небольшой силой, то верхняя нить оборвется (точка 1 на рис.\ref{inertia-3}) и груз упадет.
Это объясняется тем, что действие небольшой силы \textbf{F} за счет продолжительности взаимодействия, т.е. большого $ \Delta t $, приводит к тому, что телу сообщается существенный импульс силы \textbf{F}$ \Delta t $.
За рассматриваемый промежуток времени скорость, а, соотвественно, импульс тела успевает измениться. 
Это приводит к небольшому смещению тела, из-за чего в верхней нити возникает дополнительная деформация.
Благодаря тому, что в опыте используется тонка нить, деформации быстро достигают предела прочности и верхняя нить обрывается. 

\newpage
\subsection*{\underline{Теория:}}

В рассматриваемом эксперименте движение гири определяется следующим уравнением (второй закон Ньютона в векторной форме) 
\begin{equation}\label{inertia-eq6}
	\textbf{F}_{\text{н}} + \textbf{F}_{\text{в}} + m\textbf{g} = m\textbf{a},
\end{equation}
где $ \textbf{F}_{\text{н}} $, $ \textbf{F}_{\text{в}} $ — векторы сил натяжения нижней и верхней нити соответственно, $ m\textbf{g} $ — вектор силы тяжести, $ \textbf{a} $ — ускорение груза.

\begin{figure}[H] 
	\centering 	
	\includegraphics[width=0.4\linewidth]{inertia-3.png}
	\caption{Схематичное изображение подвешенного на нити груза}
	\label{inertia-3}
\end{figure}

При медленном натяжении имеет место «статическое» распределение сил (ускорение мало или $ a \longrightarrow 0$), проекции которых на ось $ y $ связаны следующим соотношением:
\begin{equation}\label{inertia-eq7}
F_{\text{н}} - F_{\text{в}} = mg,\\
F_{\text{в}} > F_{\text{н}}
\end{equation}
то есть сила натяжения верхней нити всегда превышает силу натяжения со стороны нижней нити на величину $ mg $.

Для быстрого натяжения имеем 
\begin{equation}\label{inertia-eq8}
F_{\text{в}} - F_{\text{н}} =  mg + ma
\end{equation}
или $ F_{\text{в}} > F_{\text{н}} $, поэтому в первую очередь обрывается верхняя нить.

Из формулы (\ref{inertia-eq8}) видно, что при значительном ускорении $ a $ (резкий рывок) для сил натяжения справедливо условие: $ F_{\text{н}} > F_{\text{в}}  $.
Отсюда следует, что разница в обрыве нижней или верхней нитей обусловливается присутствием в системе тела большой массы, подвешенного за верхнюю нить.
Таким образом, в случае резкого рывка смещение гири в силу ее инертности оказывается малым, поэтому характерное время растяжения верхней нити значительно превышает время растяжения нижней.
Поэтому для нижней нити разрывное натяжение «наступает» раньше.

\end{document}